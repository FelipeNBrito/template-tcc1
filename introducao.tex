\section{INTRODUÇÃO}

Inicia-se contextualizando o tema do trabalho e considerando os seguintes aspectos no desenvolvimento da introdução:

\begin{alineascomponto}
    \item O que o projeto enfoca? \textbf{Problema}(s) a solucionar ou equacionar, com informações sobre ele(s).
    \item O projeto atende a quem? \textbf{Público-alvo} a ser beneficiado com a ação. Deve-se descrever as características socioeconômicas, educacionais, culturais e outras que se julgar importante do público-alvo.
    \item Justificativa no presente – o projeto existe por quê? Qual a \textbf{relevância} do projeto? qual a influência que a ação proposta no projeto pode exercer na vida do público-alvo?
    \item Em alguns trabalhos, expõe-se as consequências no médio/longo prazo –  o projeto contribui para quê? Impacto do projeto as transformações positivas e duradouras esperadas.
\end{alineascomponto}

A introdução deve necessariamente contextualizar o trabalho no conhecimento atual do seu tema. Assim, deve-se citar brevemente o que outras pessoas tem feito de similar ao  trabalho proposto, acrescentando suas similaridades e diferenças com elas. Essa apresentação nesta seção da introdução é breve o suficiente para justificar a existência do seu trabalho, respondendo: de que forma ele se diferencia do que já existe? Apresentação detalhada é feita na seção “2. Trabalhos Relacionados”.

Todo o texto deve ser escrito no modo impessoal.

Quanto à formatação do texto, deve-se observar que a numeração de páginas começa a contar após a capa, e começa a ser exibida apenas na introdução.